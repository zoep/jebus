\documentclass[xcolor=table]{beamer}
\usetheme{Dresden} % My favorite!
\usecolortheme{beaver} % Simple and clean template
\usepackage{amssymb,latexsym}
\usepackage{xltxtra, xgreek}
\usepackage{graphicx}
\usepackage{textcomp}
\usepackage{url} 

\setsansfont[Mapping=tex-text]{CMU Concrete}
\setmonofont{Courier New}
\newcommand{\tab}{\hspace*{1.2em}}

\definecolor{links}{HTML}{E4287C}
\hypersetup{colorlinks,linkcolor=,urlcolor=links}
\definecolor{gray}{gray}{0.89}

%\usetheme{Boadilla} % Pretty neat, soft color.
%\usetheme{default}
%\usetheme{Warsaw}
%\usetheme{Bergen} % This template has nagivation on the left
%\usetheme{Frankfurt} % Similar to the default 
%with an extra region at the top.
%\usecolortheme{seahorse} % Simple and clean template
%\usetheme{Darmstadt} % not so good
% Uncomment the following line if you want %
% page numbers and using Warsaw theme%
% \setbeamertemplate{footline}[page number]
\setbeamercovered{transparent}
%\setbeamercovered{invisible}
% To remove the navigation symbols from 
% the bottom of slides%
\setbeamertemplate{navigation symbols}{} 
\expandafter\def\expandafter\insertshorttitle\expandafter{%
  \insertshorttitle\hfill\insertframenumber\,/\,\inserttotalframenumber}
  

 
\newcommand{\bs}[1]{\boldsymbol{#1}}  
\renewcommand{\figurename}{Figure \arabic{figure}}

%\usepackage{bm}         % For typesetting bold math (not \mathbold)
%\logo{\includegraphics[height=0.6cm]{yourlogo.eps}}
%
\title[The Jebus Lambda Calculus Interpreter]{Υλοποίηση ενός διερμηνέα για Λάμβδα Λογισμό}
\author{Ζωή Παρασκευοπούλου \\ \and Νίκος Γιανναράκης  }

\institute[]
{
Σχολή Ηλεκτρολόγων Μηχανικών και Μηχανικών Υπολογιστών\\
Εθνικό Μετσόβιο Πολυτεχνείο
}
\date{\today}
\begin{document}

\begin{frame}
\titlepage
\end{frame}

%% Dependencies
% wl-pprint
% parsec
% cmdargs

\begin{frame}
\frametitle{The Syntax}
\begin{block}
{
\begin{small}
\begin{align*}
e & := e_1 \; e_2 																		                                  & op &:= \bs{``+"}\; | \; \bs{``-"}  \\
   & | \; \bs{``\lambda"} \; id \; \bs{`` ."} \; e 										                  &        & |\; \bs{``*"} \; 	  \\
   & | \; \bs{``let"} \; [\bs{``rec"}] \; id \; \bs{``="} \; e_1 \; \bs{``in"} \; e_2 		  & 		& | \; \bs{``**"}		              \\
   & | \;  \bs{``["}e1\bs{``,"} \; e2\bs{``]"} 													              & rop & := \bs{``="} | \bs{``<"}\\
   & | \;  id 																			 	                                  & 		& | \bs{``<="} | \bs{``>"}\\
   & | \;  \bs{``true"}\; | \;\bs{``false"} 													                  &		& | \bs{``>="} \\
   & | \; \bs{``if"}\; \; e_1\; \bs{``then"}\; e_2\; \bs{``else"}\; e_3    & bop & := \bs{``\&\&"} \\
   & | \; e_1 \; op \; e_2                                                                                         &             & | \bs{``||"} \\
   & | \; e_1 \; rop \; e_2                                                                                          & \\
   & | \; e_1 \; bop \; e_2
\end{align*}
\end{small}

}

\end{block}
\end{frame}

%


\begin{frame}
\frametitle{Type System}
\begin{block}
{
The language is strongly typed featuring the \textbf{Hindley-Milner} type system. The types are implicit in the source (à la Curry) and they are automatically reconstructed using the \textbf{algorithm W} for type inference.
}
\end{block}


\end{frame}


\begin{frame}
\frametitle{Hindley-Milner typing}
\begin{block}
{ 
Hindley-Milner type system is a restriction of system F, featuring let polymorphism. Unlike system F, in which type reconstruction is undecidable, 
the types can be inferred using the algorithm W.
\\
\textbf{Significant Limitation: }
Let-polymorphism is rank-1 polymorphism, that means that functions cannot take as arguments polymorphic functions.
}

\end{block}
\end{frame}

\begin{frame}
\frametitle{Hindley-Milner typing}
\framesubtitle{Examples}
\begin{block}
{ 
     \begin{columns}[t] % contents are top vertically aligned
     \begin{column}[T]{5cm} % each column can also be its own environment
          \begin{figure}[h!]
				 \begin{small}
				 \rowcolors{1}{gray}{gray}
				 \begin{tabular}{l}
				 \textcolor{black}{
			      $>$ ./jebus annot} \\
			      \textcolor{black}{let const = \textbackslash x. \textbackslash y. x in} \\ \relax
			      \textcolor{black}{[const 1 true, const false 42]} \\
			      \textcolor{black}{-----------------------------} \\
			      \textcolor{black}{let const : a1 -> a2 -> a1 =} \\
			      \textcolor{black}{\tab \textbackslash x. : a1. \textbackslash y. : a6 . x}  \\
			      \textcolor{black}{in} \\ \relax
			      \textcolor{black}{\tab [const 1 true, const false 42]}
			      \end{tabular}	
				 \end{small}
			  \caption{Here \textit{const} has type $\forall a. \forall b. (a \rightarrow b \rightarrow a)$}
		 \end{figure}
     \end{column}
     \begin{column}[T]{5cm} % alternative top-align that's better for graphics
		 \begin{figure}[h!]
				\begin{small}
				\rowcolors{1}{gray}{gray}
			  	\begin{tabular}{l}
			      \textcolor{black}{$>$ ./jebus annot} \\
			      \textcolor{black}{let id = \textbackslash x.  x in}\\ \relax
			      \textcolor{black}{let f = \textbackslash g. [g 1, g true] in}\\
			      \textcolor{black}{\tab f id} \\
			      \textcolor{black}{Could not match type Nat with} \\
			      \textcolor{black}{type Bool}
				\end{tabular}	
			  \end{small}			  
			  \caption{g's type cannot be a polymorphic function!}
		 \end{figure}	          
     \end{column}
     \end{columns}
}

\end{block}
\end{frame}

%\frac{Gamma, x:\tau_1 \vdash e:\tau_2}{\Gamma \vdash \lambda x. e:\tau_1 \rightarrow \tau_2}var 
%\frac{}{\Gamma , x: \tau \vdash x:\tau }var  
 
\begin{frame}
\frametitle{Hindley-Milner typing}
\framesubtitle{Types}
\begin{block}{We will use $\tau$ for simple types, $\sigma$ for type schemes and $\alpha$ for type variables.}
\begin{align*}
\tau & := \tau_1\rightarrow \tau_2  &  \sigma  & := \forall  \alpha. \sigma_1\\
        & | \tau_1 \times \tau_2          &                      & | \;\tau \\
        & | Nat                                    & \\
        & | Bool                                  & \alpha & := \alpha _1 | \alpha _2| .... \\
        & | \alpha                                       & \\
\end{align*}
\end{block} 
\end{frame}  
 
\begin{frame}
\frametitle{Hindley-Milner typing}
\framesubtitle{Typing Rules}

\begin{small}
\begin{align*}
& \frac{}{\Gamma , x: \sigma \vdash x:\sigma }var  
& \frac{\Gamma, x:\tau_1 \vdash e:\tau_2}{\Gamma \vdash \lambda x. e:\tau_1 \rightarrow \tau_2}\lambda \\[0.3cm]
& \frac{\Gamma \vdash e_1:\tau_1\rightarrow\tau_2 \;\; \Gamma \vdash e_2:\tau_1}{\Gamma \vdash e_1 \; e_2:\tau_2 }@ 
& \frac{\Gamma \vdash e_1:\sigma \;\;\Gamma, x:\sigma \vdash e_2:\tau}{\Gamma \vdash let \; [rec] \; x \; = e_1 \; in \; e_2:\tau }let \\[0.3cm]
& \frac{\Gamma \vdash e: \sigma \;\; \alpha \not \in FV(\Gamma)}{\Gamma \vdash e: \forall a. \sigma }gen 
& \frac{\Gamma \vdash e: \forall a. \sigma }{\Gamma \vdash e:\sigma [\alpha \rightarrow \tau ] }inst \\[0.3cm]
& \frac{\Gamma \vdash e : Bool \;\; \Gamma \vdash e_1:\tau \;\; \Gamma \vdash e_2:\tau}{\Gamma \vdash if \; e \; then \; e_2 \; else \;e_3 : \tau}ite
& \frac{\Gamma \vdash e_1:\tau_1 \;\; \Gamma \vdash e_2:\tau_2}{\Gamma \vdash [e_1, \; e_2]:\tau_1\times \tau_2  }pair \\[0.3cm]
\end{align*}
\end{small}
\end{frame} 

\begin{frame}
\frametitle{Hindley-Milner typing}
\framesubtitle{Typing Rules (cont.)}
\begin{small}
 \begin{gather*}
 \frac{\Gamma \vdash e_1 : Nat \;\; \Gamma \vdash e_2: Nat }{\Gamma \vdash e_1 \diamond e_2: Nat \;\;\; \diamond \in \lbrace +, -, *, /, **\rbrace}op \\[0.3cm]
 \frac{\Gamma \vdash e_1 : Nat \;\; \Gamma \vdash e_2: Nat }{\Gamma \vdash e_1 \diamond e_2: Bool \;\;\; \diamond \in \lbrace <, <=,==,>,>=\rbrace}rop \\[0.3cm]
 \frac{\Gamma \vdash e_1 : Bool \;\; \Gamma \vdash e_2: Bool }{\Gamma \vdash e_1 \diamond e_2: Bool \;\;\; \diamond \in \lbrace \&\&, || \rbrace }bop \\[0.3cm]
 \frac{\Gamma \vdash e : Bool }{\Gamma \vdash not \; e: Bool}not \\
\end{gather*}
\end{small}
\end{frame} 

%

\begin{frame}
\frametitle{The Core Language}
\begin{block}{The internal representation is actually a pretty small language. Most of the language's expressions are defined as syntactic sugar}
\begin{align*}
e & := e_1 \; e_2 	 \\
   & | \; \bs{\lambda} \; id \; \bs{ .} \; e \\
   & | \; \bs{Fix} \; e_1                                                                 
\end{align*}
\end{block}
\end{frame}

%

\begin{frame}
\frametitle{The Core Language}
\framesubtitle{Syntactic Sugar}
\begin{itemize}
\item \textbf{Integers} The integers are represented internally with church encoding\\
$$n \equiv \lambda \;s.\; \lambda \; z. \; \underbrace{s(s ... (s }_\text{n times} z)..)$$
\item \textbf{Arithmetical Operations} $e_1 + e_2 \equiv (\lambda \; x. \; \lambda \; y.\; x \; succ \; y) e_1 \; e_2$\\
$e_1 - e_2 \equiv (\lambda \; x. \; \lambda \; y.\; y \; pred \; x) e_1 \; e_2$ \\
$e_1 * e_2 \equiv (\lambda \; x. \; \lambda \; y.\; \lambda \; z. \;  x \; y \; z) e_1 \; e_2$ \\
$e_1 ** e_2 \equiv (\lambda \; x. \; \lambda \; y.\;  y \; x ) e_1 \; e_2$ \\
\end{itemize}
\end{frame}

\begin{frame}
\frametitle{The Core Language}
\framesubtitle{Syntactic Sugar}
\begin{itemize}
\item \textbf{Boolean Constants} \\
$true \equiv \lambda \; x. \; \lambda \; y.\;  x \;\;\;\;\;\;$ 
$false \equiv \lambda \; x. \; \lambda \; y.\;  y$ \\
\item \textbf{Pairs} 
$[e_1, \; e_2] \equiv \lambda \;x.\; x \; e_1 \; e_2$ \\
\item \textbf{Provided functions for pairs} \\
$fst \equiv \lambda \;x.\; x \; true$ with type $\forall \;a. \forall b. \; a \times b \rightarrow a$\\
$snd \equiv \lambda \;x.\; x \; false$ with type $\forall \;a. \forall b. \; a \times b \rightarrow b$\\
\item \textbf{Provided functions for Integers} \\
$succ \equiv \lambda \; x. \; \lambda \; s. \; \lambda \; z. \;s \; (n \; s) \; z $ with type $Nat \rightarrow Nat$\\
$iszero \equiv \lambda \; x.\; x \; (true \; false) \; true $ with type $Nat \rightarrow Bool$ \\
$pred \equiv \lambda \; x.\; snd \; (x \; next \; [0,0] )$ with type $Nat \rightarrow Nat$ \\
where $next \equiv \lambda \; x. \; [succ \; (fst\;x), \; (fst\;x)]$
\end{itemize}
\end{frame}

\begin{frame}
\frametitle{The Core Language}
\framesubtitle{Syntactic Sugar}
\begin{itemize}
\item \textbf{Boolean Operators} \\
$not \equiv \lambda \; x. \; x\; false\; true$ \\
$e_1 \&\& e_2 \equiv (\lambda \; x. \; \lambda \; y.\;  x \; y \; false) \; e_1 \; e_2$ \\
$e_1 || e_2 \equiv (\lambda \; x. \; \lambda \; y.\;  x \; true \; y) \; e_1 \; e_2$ \\
\item \textbf{Relative Operators} \\
$e_1 \leq e_2 \equiv (\lambda \; x. \; \lambda \; y.\; iszero \;(n\; pred \; m)) \; e_1 \; e_2$ \\
$e_1 < e_2 \equiv (\lambda \; x. \; \lambda \; y.\; not \;(y\; leq \; x)) \; e_1 \; e_2$ \\
$e_1 == e_2 \equiv (\lambda \; x. \; \lambda \; y.\; (y\; leq \; x) \&\&(x\; leq \; y))\; e_1 \; e_2$ \\
$e_1 \geq e_2 == e_2 \leq e_1 $ \\
$e_1 > e_2 == e_2 < e_1 $ \\
\end{itemize}
\end{frame}

\begin{frame}
\frametitle{The Core Language}
\framesubtitle{Syntactic Sugar}
\begin{itemize}
\item \textbf{Let Definitions} \\
$let \; x \; = \; e_1 \; in \; e_2 \equiv (\lambda \; x. \;e_2 )\; e_1$ \\
\item \textbf{Let rec is more tricky} 
$let\; rec \; x \; = \; e_1 \; in \; e_2 \equiv (\lambda \; x. \;e_2 )\; (\bs{Y} \; (\lambda \; x. \;e_1 ))$ \\
remember that $Y \equiv \lambda \;f. \; (\lambda\; x. \; f \; (x\; x)) \; (\lambda\; x. \; f\;(x\;x))$
Alternatively, we can add a new construct to simulate $Y$'s behavior: 
$let\; rec \; x \; = \; e_1 \; in \; e_2 \equiv (\lambda \; x. \;e_2 )\; (\bs{Fix} \; (\lambda \; x. \;e_1 ))$ \\
In both cases $e_1$ is allowed to refer to $x$. The difference is that, unlike $Y$, $\bs{Fix}$ can be typed with the following rule:
$$ \frac{\Gamma \vdash e:\tau \rightarrow \tau}{\Gamma \vdash Fix \;e:\tau }fix  $$
\end{itemize}
\end{frame}

\begin{frame}
\frametitle{The Core Language}
\framesubtitle{Evaluation Strategies}
\begin{block}{
Currently Jebus supports two different evaluation strategies: normal order and applicative order, with the former being a 
non-strict evaluation strategy and the later a strict one.
}
In general:
\begin{itemize}
\item \textbf{Normal Order} The leftmost outermost redex is always reduced first
\\
\item \textbf{Applicative Order} The leftmost innermost redex is always reduced first
\end{itemize}

Both strategies evaluate the body of an unapplied  function.

\end{block}
\end{frame}

\begin{frame}
\frametitle{Evaluation Strategies}
\framesubtitle{Normal Order}
\begin{block}{The normal order reduction will always produce a normal form, if one exists!}
\begin{gather*}
\frac{}{ (\lambda \;x.\;e_1)\;e_2\rightarrow e_1 [e_2/x]} \\[0.25cm]
\frac{e_1\;\rightarrow \;e_1'}{e_1 \; e_2 \rightarrow e_1' \; e_2} \\[0.25cm]
\frac{e\;\rightarrow\;e'}{v \; e \rightarrow v \; e'} \\[0.25cm]
\frac{e\;\rightarrow\;e'}{\lambda\;x.\;e\; \rightarrow \lambda \;x. \;e'} \\[0.25cm]
\end{gather*}

\end{block}
\end{frame}


\begin{frame}
\frametitle{Evaluation Strategies}
\framesubtitle{Applicative Order}
\begin{block}{Applicative order reduction is not normalizing!}
\begin{gather*}
\frac{}{ (\lambda \;x.\;v_1)\;v_2\rightarrow v_1 [v_2/x]} \\[0.3cm]
\frac{e_1\;\rightarrow \;e_1'}{e_1 \; e_2 \rightarrow e_1' \; e_2} \\[0.3cm]
\frac{e\;\rightarrow\;e'}{v \; e \rightarrow v \; e'} \\[0.3cm]
\frac{e\;\rightarrow\;e'}{\lambda\;x.\;e\; \rightarrow \lambda \;x. \;e'} \\[0.3cm]
\end{gather*}
\end{block}
\end{frame}

\begin{frame}
\frametitle{Evaluation Strategies}
\framesubtitle{Semantics for fix}
\begin{block}{We can think fix as function that takes a function and computes its fixed point.}
\begin{gather*}
\frac{}{ (fix \; \lambda \;x.\;e) \rightarrow e[fix \; \lambda \; x. e/x]} \\[0.3cm]
\frac{e\;\rightarrow\;e'}{fix \; e \rightarrow fix \; e'} \\[0.3cm]
\end{gather*}

Note that $e[fix \; \lambda \; x. e/x] \equiv_\beta (\lambda\; x. e) \; (fix \; \lambda \; x. e) \equiv fix \; \lambda \; x. e$ ,
just like $f \;(Y\; f) \equiv Y \; f$.
\end{block}
\end{frame}


\begin{frame}
\frametitle{Evaluation Strategies}
\framesubtitle{Fix: Example}

\begin{tiny}
\begin{align*}
 & let \; rec \; fact \; = \; \lambda \; x. \; if \; iszero \; x \; then \; 1 \; else \; x * fact \; (x-1) \; in \; fact \; 3 \\
 \rightarrow & (\lambda \; fact. \;fact \; 3 )\; (\bs{fix} \; (\lambda \; fact. \lambda \; x. \; if \; iszero \; x \; then \; 1 \; else \; x * fact \; (x-1))) \\
 \rightarrow & (\bs{fix} \; (\lambda \; fact. \lambda \; x. \; if \; iszero \; x \; then \; 1 \; else \; x * fact \; (x-1))) \; 3\\
\rightarrow & (\lambda \; x. \; if \; iszero \; x \; then \; 1 \; else \; x * (\bs{fix} \; (\lambda \; fact. \lambda \; x. \; if \; iszero \; x \; then \; 1 \; else \; x * fact \; (x-1))) \; (x-1)) \; 3\\ 
\rightarrow & if \; iszero \; 3 \; then \; 1 \; else \; 3 * (\bs{fix} \; (\lambda \; fact. \lambda \; x. \; if \; iszero \; x \; then \; 1 \; else \; x * fact \; (x-1))) \; (3-1) \\
\rightarrow &  3 * (\bs{fix} \; (\lambda \; fact. \lambda \; x. \; if \; iszero \; x \; then \; 1 \; else \; x * fact \; (x-1))) \; 2 \\
\rightarrow &  3 * (\lambda \; x. \; if \; iszero \; x \; then \; 1 \; else \; x * (\bs{fix} \; (\lambda \; fact. \lambda \; x. \; if \; iszero \; x \; then \; 1 \; else \; x * fact \; (x-1))) \; (x-1)) \; 2 \\
\rightarrow & 3 * if \; iszero \; 2 \; then \; 1 \; else \; 2 * (\bs{fix} \; (\lambda \; fact. \lambda \; x. \; if \; iszero \; x \; then \; 1 \; else \; x * fact \; (x-1))) \; (2-3) \\
\rightarrow & 3 * 2 * (\bs{fix} \; (\lambda \; fact. \lambda \; x. \; if \; iszero \; x \; then \; 1 \; else \; x * fact \; (x-1))) \; 1 \\
\rightarrow & 3 * 2 *(\lambda \; x. \; if \; iszero \; x \; then \; 1 \; else \; x * (\bs{fix} \; (\lambda \; fact. \lambda \; x. \; if \; iszero \; x \; then \; 1 \; else \; x * fact \; (x-1))) \; (x-1)) \; 1 \\
\rightarrow & 3 * 2 * if \; iszero \; 1 \; then \; 1 \; else \; 1 * (\bs{fix} \; (\lambda \; fact. \lambda \; x. \; if \; iszero \; x \; then \; 1 \; else \; x * fact \; (x-1))) \; (1-1) \; \\
\rightarrow & 3 * 2 * 1 * (\bs{fix} \; (\lambda \; fact. \lambda \; x. \; if \; iszero \; x \; then \; 1 \; else \; x * fact \; (x-1))) \; 0 \; \\
\rightarrow & 3 * 2 * 1 * (\lambda \; x. \; if \; iszero \; x \; then \; 1 \; else \; x * (\bs{fix} \; (\lambda \; fact. \lambda \; x. \; if \; iszero \; x \; then \; 1 \; else \; x * fact \; (x-1))) \; (x-1)) \; 0 \; \\
\rightarrow & 3 * 2 * 1 * if \; iszero \; 0 \; then \; 1 \; else \; 0 * (\bs{fix} \; (\lambda \; fact. \lambda \; x. \; if \; iszero \; x \; then \; 1 \; else \; x * fact \; (x-1))) \; (0-1) \\
\rightarrow & 3 * 2 * 1 *  1  \\
\end{align*}
\end{tiny}

\end{frame}


\begin{frame}
\frametitle{Evaluation Strategies}
\framesubtitle{Normal Order vs. Applicative Order}
\begin{block}{Consider the following programs:
  \begin{columns}[t] % contents are top vertically aligned
     \begin{column}[T]{5cm} % each column can also be its own environment
          \begin{figure}[h!]
				 \begin{footnotesize}
				 \rowcolors{1}{gray}{gray}
				 \begin{tabular}{l}
				 \textcolor{black}{
			      $>$ cat ite.lam} \\
			      \textcolor{black}{let f = \textbackslash x.} \\ 
			      \textcolor{black}{\tab  if (iszero x) then x + 3} \\
				  \textcolor{black}{\tab  else x * 3} \\
			      \textcolor{black}{in} \\
			      \textcolor{black}{\tab f 0} \\
			      \end{tabular}	
				 \end{footnotesize}
			  \caption{ite.lam}
		 \end{figure}
     \end{column}
     \begin{column}[T]{5cm} % alternative top-align that's better for graphics
		 \begin{figure}[h!]
				\begin{footnotesize}
				\rowcolors{1}{gray}{gray}
			  	\begin{tabular}{l}
			  	 \textcolor{black}{
			      $>$ cat fact.lam} \\
			      \textcolor{black}{let rec fact = \textbackslash x. } \\ 
			      \textcolor{black}{\tab if (iszero x) then 1} \\ 
			      \textcolor{black}{\tab else x * fact (x-1)} \\
			      \textcolor{black}{in} \\
			      \textcolor{black}{\tab fact 4} \\
				\end{tabular}	
			  \end{footnotesize}			  
			  \caption{fact.lam}
		 \end{figure}	          
     \end{column}
     \end{columns}
}
\end{block}
\end{frame}

\begin{frame}
\frametitle{Evaluation Strategies}
\framesubtitle{Normal Order vs. Applicative Order}
Applicative order needs 4 more beta reductions.  
Applicative order is a strict reduction strategy so both the then and the else parts
will be evaluated.
\begin{block}{
  \begin{columns}[T] % contents are top vertically aligned
     \begin{column}[T]{6cm} % each column can also be its own environment
          \begin{figure}[h!]
				 \begin{footnotesize}
				 \rowcolors{1}{gray}{gray}
				 \begin{tabular}{l}
				 \textcolor{black}{
			      $>$ ./jebus eval -e=normal -t < ite.lam} \\
			      \textcolor{black}{(\textbackslash f . f (\textbackslash f . \textbackslash x . x)) $\ldots \ldots$ =>} \\ 
			      \textcolor{black}{$\ldots$ =>} \\
				  \textcolor{black}{$\ldots$} \\
			      \textcolor{black}{$\ldots$ =>} \\
			      \textcolor{black}{\textbackslash f . \textbackslash x . f (f (f x))} \\
			      \textcolor{black}{Performed 11 beta reductions.} \\
			      \end{tabular}	
				 \end{footnotesize}
			  \caption{Evaluate ite.lam with normal order strategy. Only 11 beta reductions needed.}
		 \end{figure}
     \end{column}
     \begin{column}[T]{6cm} % alternative top-align that's better for graphics
		 \begin{figure}[h!]
				\begin{footnotesize}
				\rowcolors{1}{gray}{gray}
				 \begin{tabular}{l}
				 \textcolor{black}{
			      $>$ ./jebus eval -e=applicative -t < ite.lam} \\
			      \textcolor{black}{(\textbackslash f . f (\textbackslash f . \textbackslash x . x)) $\ldots \ldots$ =>} \\ 
			      \textcolor{black}{$\ldots$ =>} \\
				  \textcolor{black}{$\ldots$} \\
			      \textcolor{black}{$\ldots$ =>} \\
			      \textcolor{black}{\textbackslash f . \textbackslash x . f (f (f x))} \\
			      \textcolor{black}{Performed 14 beta reductions.} \\
			      \end{tabular}	
			  \end{footnotesize}			  
			  \caption{Evaluate ite.lam with applicative order strategy. 15 beta reductions needed.}
		 \end{figure}	          
     \end{column}
     \end{columns}
}

\end{block}
\end{frame}


\begin{frame}
\frametitle{Evaluation Strategies}
\framesubtitle{Normal Order vs. Applicative Order}
\begin{block}{
          \begin{figure}[h!]
				 \begin{footnotesize}
				 \rowcolors{1}{gray}{gray}
				 \begin{tabular}{l}
				 \textcolor{black}{
			      $>$ ./jebus eval -e=normal -t < fact.lam} \\
			      \textcolor{black}{((\textbackslash . fac (\textbackslash f . \textbackslash x . f (f (f (f x))))) $\ldots \ldots$ =>} \\ 
			      \textcolor{black}{$\ldots$ =>} \\
				  \textcolor{black}{$\ldots$} \\
			      \textcolor{black}{$\ldots$ =>} \\
			      \textcolor{black}{\textbackslash z . \textbackslash x . z (z (z (z (z (z (z (z (z (z (z (z (z (z (z (z (z (z (z (z (z (z (z (z x)...)} \\
			      \textcolor{black}{Performed 9236 beta reductions.} \\
			      \end{tabular}	
				 \end{footnotesize}
			  \caption{Evaluate fact.lam with normal order strategy. The program terminates after 9236 reductions.}
		 \end{figure}    
}

\end{block}
\end{frame}

\begin{frame}
\frametitle{Evaluation Strategies}
\framesubtitle{Normal Order vs. Applicative Order}
\begin{block}{
		 \begin{figure}[h!]
				\begin{footnotesize}
				\rowcolors{1}{gray}{gray}
				 \begin{tabular}{l}
				 \textcolor{black}{
			      $>$ ./jebus eval -e=applicative -t < fact.lam} \\
			      \textcolor{black}{((\textbackslash . fac (\textbackslash f . \textbackslash x . f (f (f (f x))))) $\ldots \ldots$ =>} \\ 
			      \textcolor{black}{$\ldots$ =>} \\
				  \textcolor{black}{$\ldots$} \\
			      \textcolor{black}{$\ldots$} \\
			      \end{tabular}	
			  \end{footnotesize}			  
			  \caption{Evaluate fact.lam with applicative order strategy. The program does not terminate.}
		 \end{figure}	          
}
\end{block}
\end{frame}

\begin{frame}
\frametitle{How to use Jebus}
\begin{scriptsize}
Jebus reads a program from the standard input and can print the type annotated version of the program after the type inference or evaluate the program with the selected strategy. You can also trace the evaluation and count the number of reduction steps. \\ \medskip
\rowcolors{1}{gray}{gray}
\begin{tabular}{l p{5cm}}
\multicolumn{2}{l}{jebus [COMMAND] ... [OPTIONS]} \\
&\\
Common flags: &\\
 \tab -h --help &            Display help message \\
 \tab -V --version &       Print version information \\
&\\
jebus annot& \\
\multicolumn{2}{l}{\tab  Print an explicitly typed version of the program}\\
&\\
jebus eval [OPTIONS] &\\
\multicolumn{2}{l}{\tab  Interpret the program} \\
&\\
 \tab  -t --trace         & show each beta reduction \\
 \tab -e --eval=EVALMODE & specify evaluation strategy: normal (default) or applicative \\
                                    

\end{tabular}
\end{scriptsize}
\end{frame}


\begin{frame}
\frametitle{Useful links}
\begin{block}
{
\begin{footnotesize}
\begin{itemize}
\item \href{http://www.math.ntua.gr/logic/lambda/notes/shmeiwseis-2011.pdf}{Notes} form NTUA's Applications of Logic in Computer Science course %
\item \href{http://homepage.cs.uiowa.edu/~slonnegr/plf/Book/Chapter5.pdf}{Chapter 5} from the book Formal Syntax and Semantics of Programming Languages, Kenneth Slonneger, Barry L. Kurtz
\item \href{http://www.cs.uu.nl/wiki/pub/Cco/CourseSchedule/cco-09-typesystems-2x2.pdf}{Hindley-Milner Typing and Algorithm W} from Compiler Construction course notes, Utrech University %
\item \href{http://lambda.jimpryor.net/lambda_library/}{lambda library} from NYU Lambda Seminar %
\item \href{http://homes.cs.washington.edu/~djg/2011sp/lec11_6up.pdf}{Simply typed lambda calculus extensions} from Programming Languages course notes, University of Washington %
\end{itemize}
\end{footnotesize}
}

\end{block}
\end{frame}

\begin{frame}
\frametitle{The end!}
\begin{block}
{\begin{center}
{\large Demo}
\end{center}}
\begin{center}
 \href{https://github.com/zoep/jebus}{Fork here!}
 \end{center} 
\end{block}
\end{frame}
% End of slides
\end{document} 
